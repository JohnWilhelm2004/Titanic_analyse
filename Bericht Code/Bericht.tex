\documentclass[a4paper, 12pt]{article}

% --- Sprache und Kodierung ---
\usepackage[utf8]{inputenc}       % Ermöglicht Umlaute direkt einzugeben
\usepackage[T1]{fontenc}          % Korrekte Trennung von Wörtern mit Umlauten
\usepackage[ngerman]{babel}       % Deutsche Silbentrennung und Bezeichnungen (z.B. "Abbildung")
\usepackage{csquotes}             % Für korrekte deutsche Anführungszeichen

% --- Seitenlayout und Ränder ---
% Hinweis: Deine Titelseite hatte sehr schmale Ränder. Für den Lesetext
% sind 2,5cm Standard. Wir können für die Titelseite später Ausnahmen definieren.
\usepackage[left=2.5cm, right=2.5cm, top=2.5cm, bottom=2.5cm]{geometry}
\usepackage{setspace}             % Für Zeilenabstand
\onehalfspacing                   % 1.5-facher Zeilenabstand (Standard für Berichte)
\usepackage{pdfpages}

% --- Schriftart (Optional) ---
% Standard ist Computer Modern. Viele Berichte nutzen Times (mathptmx) oder Palatino (mathpazo).
% Wenn du den Standard magst, kannst du die nächste Zeile auskommentieren.
\usepackage{lmodern}              % Eine schärfere Version der Standardschrift

% --- Grafiken und Bilder ---
\usepackage{graphicx}             % Zum Einbinden von Bildern (deine R-Plots)
\usepackage{float}                % Erlaubt die Positionierung [H] (Hier und wirklich hier)
\usepackage{caption}              % Schöneren Bildunterschriften
\usepackage{subcaption}           % Erlaubt mehrere Bilder nebeneinander (a, b)

% --- Tabellen (aus deiner Titelseite übernommen & erweitert) ---
\usepackage{booktabs}             % Für professionelle Tabellenlinien (\toprule, \midrule)
\usepackage{rotating}             % Um Tabellen zu drehen
\usepackage{tabularx}             % Tabellen mit automatischer Spaltenbreite
\usepackage{multirow}             % Zellen über mehrere Zeilen verbinden
\usepackage{color, colortbl}      % Farben in Tabellen
\definecolor{Gray}{gray}{0.9}     % Deine definierte Farbe

% --- Mathematik und Symbole ---
\usepackage{amsmath}              % Wichtig für statistische Formeln (Korrelationen etc.)
\usepackage{amssymb}              % Mathematische Symbole
\usepackage{amsfonts}

% --- Verweise und Links ---
\usepackage[hidelinks]{hyperref}  % Klickbare Links im PDF (ohne rote Kästen)

% --- Blindtext (nur zum Testen, kann später raus) ---
\usepackage{blindtext}

\begin{document}
\begin{titlepage}
	% Ränder temporär anpassen
	\newgeometry{left=2cm, right=2cm, top=2cm, bottom=2cm}

	\begin{center}
		\Large
		Technische Universität Dortmund\\
		Wintersemester 2025/26\\
		
		\vfill % Füllt den Platz dynamisch bis zum nächsten Element
		
		Wissenschaftliches Arbeiten: Bericht über Titanic-Datensatz
		
		\vspace{1em} % Ein kleiner fester Abstand zwischen Untertitel und Titel sieht meist besser aus
		
		\Huge
		\textbf{Einflussfaktoren auf die Überlebenschancen beim Untergang der Titanic}
		
		\vfill
		
		% \Large
		% \textbf{DozentInnen:}\\
		% ...
		
		% \vfill
		
		\textbf{Verfasser:}\\
		Lukas König\\
		Elaha Bahir\\
		John Wilhelm\\
        Annika Homm
		
		\vfill
		
		06.02.2026
	\end{center}
	
	\restoregeometry
\end{titlepage}

\section{Einleitung}
Der Untergang der Titanic ist ein weltbekanntes Ereignis in der Geschichte der Menschheit. ES ist Gegenstand zahlreicher historischer und statistischer Forschung. In diesem Bericht soll es darum gehen, durch eine explorative Datenanalyse signifikante Merkmale (wie Geschlecht, Alter, Klasse) der Passagiere auf der Titanic zu identifizieren und Zusammenhänge zum Überleben zu finden.

\section{Daten und Methoden}
\begin{itemize}
    \item \textbf{Daten:} Zur Datenanalyse wurde der Datensatz titanic.csv und die Pakete "tidyverse" und "ggplot2" verwendet. Vor der Datenbereinigung wies die Variable \textit{Age} 177 NA-Werte (ca. 19,86\%) auf, \textit{Embarked} besaß zwei leere Zellen (ca. 0,22\%) und \textit{Cabin} hatte 687 leere Zellen, was einer Quote von ca. 77,10\% entspricht.
    \item \textbf{Methodik:}
    Leere Zeichenketten wurden als \texttt{NA} kodiert und nicht für die Analyse relevante Variablen (wie \textit{PassengerID, Name, Ticket, Cabin}) wurden nach Merkmalsextraktion entfernt.
    \begin{itemize}
        \item \textbf{Age (Alter in Jahren beim Untergang):} Fehlende Werte wurden durch das arithmetische Mittel der jeweiligen Titelgruppe ersetzt, um die demografische Struktur beizubehalten und keinen Bias zu kreieren.
        \item \textbf{Embarked (Zustiegshafen):} Zwei fehlende Einträge wurden durch den Modus "Southhampton" (S) ersetzt.
        \item \textbf{Cabin (Kabinennummer):} Aus der Kabinennummer wurden das \textbf{Deck} (\textit{Deck}) und die \textbf{Schiffsseite} (\textit{Side}) mittels Modulo-Operation extrahiert.
        \item \textit{Hinweis: Aufgrund der hohen NA-Quote der Variablen \textit{Deck} und \textit{Side} und da sie durch Passagiere der dritten Klasse überrepräsentiert (ca. 69,72\%) ist, ist eine Analyse dieser Variablen verzerrt (Selection Bias). Im weiteren Verlauf des Berichts wird daher de Fokus auf die robusteren Variablen wie Geschlecht, Klasse und Alter gelegt.}
    \end{itemize}
    \item \textbf{Datentransformation:}
    Die Variablen \textit{Sex, Survived, Embarked, Side} und \textit{Deck} wurden als Faktoren kodiert. Die Passagierklasse (\textit{Pclass}) wurde als \textbf{ordered factor} definiert umd die Rangfolge abzubilden.
\end{itemize}

\section{Ergebnisse}

%Plot0% 
\subsection{Zusammenhang zwischen katergoriellen Merkmalen und Überlebensstatus}
%Grafik Plot 0%
\begin{figure}[H]
    \centering
    \includegraphics[width=0.9\textwidth]{Plot0.pdf}
    \caption{Abbildung 1: Absolute Häufigkeiten kategorieller Daten gruppiert nach Überlebensstatus}
\end{figure}

% Plot 0: Absolute Häufigkeiten kategorieller Daten, und davon die relative Häufigkeiten der Überlebten
% - Diese Gruppen waren am meisten vertreten (Demografie) (dritte Preisklasse, Männer, die  meisten haben nicht überlebt)
% - Diese Gruppen haben am ehesten überlebt (Frauen, Erste Preisklasse)
% - Beim Geschlecht war der Unterschied am größten
%    -> weil: Frauen und Kinder zuerst?
%    -> weil: Männer waren meist allein

Die Demografie der Passagiere am Bord der Titanic bestand zum Großteil aus Männern und der Passagierklasse 3. Mehr als die Hälfte, ca. 61,61\% , haben den Untergang der Titanic nicht überlebt. In der ersten Passagierklasse, hat die Mehrheit überlebt, in der zweiten Passagierklasse war es fast ausgeglichen und in der dritten Klasse ist die große Mehrheit gestorben. Interessanterweise überlebten in jeder Klasse fast gleich viele, doch der Anteil der Menschen in der dritten Klasse ist weitaus größer als in den anderen zwei Klassen, die von der Größe her vergleichbar sind, wobei Klasse 1 noch etwas mehr vertreten ist. Ein weiterer großer Unterschied besteht zwischen den überlebten Frauen und Männern, denn Unter den Überlebenden gab es ca. doppelt so viele Frauen wie Männer. Der Anteil der Überlebenden bestand aus ca. 68,12\% aus Frauen. Dem gegenüber gab es fast doppelt so viele Männer an Bord als Frauen. Unter den Männern überlebten nur ca. 31,87\%.



%Plot1% 
\subsection{Zusammenhang zwischen Ticketpreisen und Überleben}
%Grafik Plot 1%
\begin{figure}[H]
    \centering
    \includegraphics[width=0.9\textwidth]{Plot1.pdf}
    \caption{Abbildung 2: Boxplot der Ticketpreise gruppiert nach Überlebensstatus}
\end{figure}

Es gibt bei den Verstorbenen eine deutliche Tendenz nach unten, der Median orientiert sich an das untere Quartil und beide Quartile liegen unter den Quartilen der Überlebenden. Zudem gibt es bei den Verstorbenen auch viele aureißer nach oben, was darauf hindeutet, dass vereinzelt auch Passagiere mit teureren Tickets verstorben sind, allerdings verhältnismäßig wenige. Bei den Überlebten liegt der mediane Ticketpreis knapp unter 30 Währungseinheiten und somit fast 3 mal teurer als der mediane Ticketpreis der Verstorbenen (ca. 10 Währungseinheiten). Der Median liegt zudem mittig und es gibt keine Ausreißer.


%Plot2% 
\subsection{Zusammenhang zwischen Familiengröße und Überleben}
%Grafik Plot 2%
\begin{figure}[H]
    \centering
    \includegraphics[width=0.9\textwidth]{Plot2.pdf}
    \caption{Abbildung 3: Balkendiagramm der Familiengröße gruppiert nach Überlebnsstatus}
\end{figure}

% Plot 2: Histogramm Familiengröße und Überleben (Kleine Familien überleben eher, vielleicht weil sie nicht alle in ein boot passen -> passt auch zur these "frauen und kinder zuerst?")

bei der Mehrheit der Passagiere handelte es sich um Alleinreisende, wobei der Großteil von ihnen nicht überlebte. Auch Familien ab einer Größe von fünf und höher sind zu 85,10\% verstorben, wobei diese eher selten vertreten waren.

%Plot3% 
\subsection{Zusammenhang zwischen Alter und Überleben}
%Grafik Plot 3%
\begin{figure}[H]
    \centering
    \includegraphics[width=0.9\textwidth]{Plot3.pdf}
    \caption{Abbildung 4: Dichteverteilungsfunktion des Alters gruppiert nach Überlebensstatus}
\end{figure}

% Plot 3: Alter und Überleben (Kinder überleben eher, bestätigt das Prinzip, ALLERDINGS mit Vorsicht zu interpretieren, da die NAs mit dem Durchschnittsalteraufgefüllt wurden)
Sowohl unter den Überlebenden, als auch den Verstorbenen liegt das Alter meist zwischen ca. 16-40 Jahren. Aufällig ist das lokale Maximum bei den Überlebenden zwischen etwas 0 und 10 jahren, welcher bei den verstorbenen nicht vorhanden ist. Dies deutet darauf hin, dass Babys und kleine Kinder deutlich häufiger überlebt haben. Ein zweites lokales Maximum befindet sich bei den Verstorbenen von etwa 30 jahren gibt, welcher methodisch bedingt (Datenimputation der Mittelwerte) und repräsentiert keine natürliche Häufung des Alters. Da die Datenimputation nach dem Mitelwert der jeweiligen Titel-klasse erfolgt ist, ist die Interpretation, dass Kinder (z.B. mit dem Titel Master) häufiger überleben nicht gefährdet.

\section{Fazit}

Zusammenfassend lässt sich sagen, dass überwiegend Frauen überlebt haben, und das obwohl sie demografisch weniger an Bord vertreten waren. Zusätzlich haben auch Kinder deutlich häufiger überlebt, was auf das Prinzip von "Frauen und Kinder zuerst" schließen lässt. Dies könnte auch eine Erklärung für die Ausreißer in Abbildung 2 sein, da es sich um reiche Männer handeln könnte, die trotz erster Klasse an Bord blieben. Große Familien haben häufig nicht überlebt, vermutlich weil sie nicht alle auf ein Rettungsboot gepasst haben und daher zurückbleiben mussten. Auch Alleinreisende sind zu einem großen Anteil verstorben, eventuell weil ca. 60,33\% von ihnen die dritte Passagierklasse besaß oder weil ca. 76,53\% von ihnen männlich waren. Letztendlich lässt sich sagen, dass vermutlich sowohl die Moral (Frauen und Kinder zuerst), als auch das Geld die Passagiere gerettet hat.

\end{document}
