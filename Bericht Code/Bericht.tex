\documentclass{article}
\usepackage{graphicx} % Required for inserting images

\title{Wissenschaftliches Arbeiten}
\author{Lukas König, Annika Homm}
\date{February 2026}

\begin{document}

\maketitle

\section{Einleitung}
Der Untergang der Titanic ist ein weltbekanntes Ereignis in der Geschichte der Menschheit. Seit jeher wirft es viele Frage auf. In diesem Bericht soll es darum gehen, welche Menschengruppen am misten auf dem Schiff vertreten waren und welche von ihnen eher überlebt haben.

\section{Ergebnisse}

Plot 1: Absolute Häufigkeiten kategorieller Daten, und davon die relative Häufigkeiten der Überlebten
- Diese Gruppen waren am meisten vertreten (dritte Preisklasse, Männer, die  meisten haben nicht überlebt)
- Diese Gruppen haben am ehesten überlebt (Frauen, Erste Preisklasse)
- Beim Geschlecht war der Unterschied am größten
   ->  weil Frauen vielleicht tendenziell mehr in der 
       hohen Ticketpreiskategorie vertreten sind? -> Korrelation zwischen Geschlecht Frau und erste Klasse
       Oder doch eher wegen dem Prinzip: Frauen und Kinder zuerst?

Plot 2: Histogramm Alter und Überleben (Kinder überleben eher, bestätigt das Prinzip)
Plot 3: Histogramm Familiengröße und Überleben (Kleine Familien überleben eher, vielleicht weil sie nicht alle in ein boot passen -> passt auch zur these "frauen und kinder zuerst?")

\section{Fazit}

Zusammenfassend lässt sich sagen, dass überwiegend ... überlebt haben ...

\end{document}
